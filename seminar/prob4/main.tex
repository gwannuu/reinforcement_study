\documentclass[8pt]{beamer}
\usefonttheme[onlymath]{serif}
% 테마 선택 (선택 사항)
\usetheme{Madrid} % 기본 테마, 다른 테마 사용 가능
% \font{serif}
\usepackage{amsfonts}
\usepackage{amssymb}

% \setcounter{MaxMatrixCols}{20}

% (필요한 패키지들)
% \usepackage{amsthm}
\setbeamertemplate{theorems}[numbered]  % 정리, 정의 등에 번호를 달아줌

% \theoremstyle{plain} % insert bellow all blocks you want in italic
% \newtheorem{theorem}{Theorem}[section] % to number according to section
% 
% \theoremstyle{definition} % insert bellow all blocks you want in normal text
% \newtheorem{definition}{Definition}[section] % to number according to section
% \newtheorem*{idea}{Proof idea} % no numbered block
\usepackage{tcolorbox}

% 필요할 경우 패키지 추가
\usepackage{graphicx} % 이미지 삽입을 위한 패키지
\usepackage{amsmath}   % 수식 사용
\usepackage{hyperref}  % 하이퍼링크 추가
\usepackage{cleveref}
\usepackage{multicol}  % 여러 열 나누기


\newcommand{\mrm}[1]{\mathrm{#1}}
\newcommand{\mbb}[1]{\mathbb{#1}}
\newcommand{\mb}[1]{\mathbf{#1}}
\newcommand{\mc}[1]{\mathcal{#1}}
\newcommand{\tb}[1]{\textbf{#1}}
\newcommand{\ti}[1]{\textit{#1}}
\newcommand{\mypois}[1]{\operatorname{Pois}(#1)}

\newcommand{\mybin}[2]{\operatorname{Bin}\!\left(#1,#2\right)}
\newcommand{\mytoinf}[1]{#1 \rightarrow \infty}
\newcommand{\myexp}[1]{\exp{\left(#1\right)}}
\newcommand{\myunif}[2]{\operatorname{Unif}\!\left(#1, #2\right)}

% 발표 제목, 저자, 날짜 설정
\title{Probability}
\author{Gwanwoo Choi}
% \date{}

\begin{document}
% 표지 슬라이드
\begin{frame}
    \titlepage
\end{frame}

% 목차 슬라이드
\begin{frame}{contents}
    \tableofcontents
\end{frame}

\begin{frame}{Continuous Random Variable}
    \begin{definition}[Continuous r.v.]
        An r.v. has a continuous distribution if its CDF is differentiable. We also allow there to be endpoints (or finitely many points) where the CDF is continuous but not differentiable, as long as the CDF is differentiable everywhere else. A continuous random variableis a random variable with a continuous distribution.
    \end{definition}

    \begin{definition}[Probability density function]
        For a continuous r.v. $X$ with CDF $F$, the probability density function (PDF) of $X$ is the derivative $f$ of the CDF, given by $f(X) = F^\prime(x)$. The support of $X$, and of its distribution, is the set of all $x$ where $f(x)>0$.
    \end{definition}

    \begin{block}{Proposition 3 (PDF to CDF)}
        Let $X$ be a continuous r.v. with PDF $f$. Then the CDF of $X$ is given by 
        \[F(x) = \int^x_{- \infty} f(t) dt\]
    \end{block}

\end{frame}

\begin{frame}{Continuous Rancom Variable}
    \begin{itemize}
        \item Note that $P(X=x) = 0, \forall x $  in continuous r.v. setting. We can discuss about probability with only integral of PDF.
        \item By the fact $P(X=x)=0, \forall x$, $P(a<X\leq b) = P(a\leq X <b) = P(a< X <b) = P(a \leq X \leq b)$
        \item Because the quantity of PDF $f(x)$ is not a probability, for some values of $x$, it is possible to have $f(x) > 1$.
        \item  Although the height of PDF directly represents the probability, it is closely related to the probability. For small $\epsilon$, 
        \[
            P(3 - \epsilon /2 < X \leq 3 + \epsilon/2 ) = \int_{3- \epsilon/2}^{3 + \epsilon/2} f(x) dx \approx f(x) dx
        \]
    \end{itemize}

    \begin{theorem}[Valid PDFs]
        The PDF $f$ of a continuous r.v. must satisfy the following two criteria:
        \begin{itemize}
            \item Nonnegative: $f(x)\leq 0$
            \item Integrates to $1$: $\int_{-\infty}^\infty f(x) dx = 1$
        \end{itemize}
    \end{theorem}
\end{frame}

\begin{frame}{Continuous Random Variable}
    \begin{example}
        \begin{itemize}
            \item Logistic 
            \[
                F(x; \mu, \sigma) = \frac{\myexp{\frac{x-\mu}{\sigma }}}{1 + \myexp{\frac{x-\mu}{\sigma }}}, \quad f(x; \mu, \sigma) = \frac{\myexp{\frac{x-\mu}{\sigma }}}{\sigma \left(1 + \myexp{\frac{x-\mu}{\sigma }}\right)^2}
            \]
            \item Rayleigh 
            \[
                F(x; \mu, \sigma) = 1  - \myexp{- \frac{(x-\mu)^2}{2 \sigma^2}}, x > \mu \quad f(x; \mu, \sigma) = \frac{x-\mu}{\sigma^2} \myexp{-\frac{(x-\mu)^2}{2 \sigma^2}}, x > 0
            \]
        \end{itemize}
    \end{example}

   

    \begin{definition}[Expectation of a continuous r.v.]
        The expected value (also called the expectation or mean) of a continuous r.v. $X$ with PDF $f$ is 
        \[
        E(X) = \int_{-\infty}^\infty x f(x) dx
        \]
    
    \end{definition}
\end{frame}

\begin{frame}{Continuous Random Variable}
    \begin{theorem}[LOTUS, continuous]
        If $X$ is a continuous r.v. with PDF $f$ and $g$ is a function from $\mathbb{R}$ to $\mathbb{R}$, then
        \[
            E[g(X)] = \int_{-\infty}^\infty g(x)f(x) dx
        \]
    \end{theorem}

    \begin{definition}[Uniform distribution]
        A continuous r.v. is said to have Uniform distribution on the interval $(a,b)$ if its PDF is
        \[
        f(x) = \begin{cases}
            \frac{1}{b-a}\quad \text{if } a<x<b, \\ 0 \quad \text{otherwise},
        \end{cases}
        \]
        This is denoted by $U \sim \myunif{a}{b} $
    \end{definition}

    PDF of uniform distribution $F(x)$ is defined as
    \[
        F(x) = \begin{cases}
            0 \quad \text{if } x\leq a,
            \\ \frac{x-a}{b-a} \quad \text{if } a < x <b,
            \\ 1 \quad \text{if } x \geq b,
        \end{cases}
    \]
\end{frame}

\begin{frame}{Continuous Random Variable}
    Expectation of uniform distribution $U \sim \myunif{a}{b}$ is simply calculated by 
    \[
    E[U] = \int_a^b \frac{x}{b-a} dx = \frac{1}{b-a} \frac{(b-a)^2}{2} = \frac{a+b}{2}
    \]

    $E[U^2]$ is calculated by using the continuous version of LOTUS
    \[
    E[U] = \int_a^b \frac{x^2}{b-a} dx = \frac{1}{b-a} \frac{1}{3}(b^3-a^3) = \frac{b^2 + ab + a^2}{3}
    \]

    By $Var[U] = E[U^2] - E[U]^2$,
    \[
        E[U] = \frac{(a-b)^2}{12}
    \]

\end{frame}


\end{document}