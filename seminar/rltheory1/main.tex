\documentclass[8pt]{beamer}
\usefonttheme[onlymath]{serif}


\setbeamertemplate{frametitle}{%
  \vskip1ex
  \usebeamerfont{frametitle}%
  \insertframetitle\par        %  ← 원하는 대로 변경 가능
  \vskip1ex
  \hrule                             % 밑줄(선택)
}

% 테마 선택 (선택 사항)
% \usetheme{Madrid} % 기본 테마, 다른 테마 사용 가능
% \font{serif}
\usepackage{amsfonts}
\usepackage{amssymb}
\usepackage[T1]{fontenc} % To use combination of textbf, textit


% \setcounter{MaxMatrixCols}{20}

% (필요한 패키지들)
% \usepackage{amsthm}
\setbeamertemplate{theorems}[numbered]  % 정리, 정의 등에 번호를 달아줌

% \theoremstyle{plain} % insert bellow all blocks you want in italic
% \newtheorem{theorem}{Theorem}[section] % to number according to section
% 
% \theoremstyle{definition} % insert bellow all blocks you want in normal text
% \newtheorem{definition}{Definition}[section] % to number according to section
% \newtheorem*{idea}{Proof idea} % no numbered block
\usepackage{tcolorbox}

% 필요할 경우 패키지 추가
\usepackage{graphicx} % 이미지 삽입을 위한 패키지
\usepackage{amsmath}   % 수식 사용
\usepackage{hyperref}  % 하이퍼링크 추가
\usepackage{cleveref}
\usepackage{multicol}  % 여러 열 나누기
\usepackage{ulem} % 취소선 및줄 나누기



\newcommand{\mrm}[1]{\mathrm{#1}}
\newcommand{\mbb}[1]{\mathbb{#1}}
\newcommand{\mb}[1]{\mathbf{#1}}
\newcommand{\mc}[1]{\mathcal{#1}}
\newcommand{\tb}[1]{\textbf{#1}}
\newcommand{\ti}[1]{\textit{#1}}
\newcommand{\mypois}[1]{\operatorname{Pois}(#1)}

\newcommand{\mybin}[2]{\operatorname{Bin}\!\left(#1,#2\right)}
\newcommand{\mytoinf}[1]{#1 \rightarrow \infty}
\newcommand{\myexp}[1]{\exp{\left(#1\right)}}
\newcommand{\myunif}[2]{\operatorname{Unif}\!\left(#1, #2\right)}
\newcommand{\mygeom}[1]{\operatorname{Geom}\!\left(#1\right)}
\newcommand{\myexpo}[1]{\operatorname{Expo}\!\left(#1\right)}
\newcommand{\abs}[1]{\left\lvert #1 \right\rvert}
\newcommand{\norm}[1]{\left\lVert #1 \right\rVert}
\newcommand{\expec}[1]{\operatorname{E}\left[ #1 \right]}
\newcommand{\myvar}[1]{\operatorname{Var}\left[#1\right]}
\newcommand{\myskew}[1]{\operatorname{Skew}\!\left[#1\right]}



% 발표 제목, 저자, 날짜 설정
\title{RL Theory}
\author{Gwanwoo Choi}
% \date{}

\begin{document}
% 표지 슬라이드
\begin{frame}
    \titlepage
\end{frame}

% % 목차 슬라이드
% \begin{frame}
%     \frametitle{Table of Contents}
%     \tableofcontents
% \end{frame}

\subsection{MDP}

\begin{frame}
    \frametitle{Table of Contents}
    \tableofcontents[currentsubsection]
\end{frame}


\begin{frame}{Markov Decision Processes}
    In MDP, transition probability $P_{s s^\prime}^a = Pr(s^\prime | s,a)$ can be expressed by matrix form $P \in \mbb{R}^{(\abs{\mathcal{S}} \cdot \abs{\mathcal{A}}) \times \mathcal{S}}$

\end{frame}


\begin{frame}{MDP}
    \begin{corollary}
        Suppose that $\pi$ is a stationary policy. We have that
        \[
        Q^\pi = (I - \gamma P^\pi)^{-1} r
        \]
        Where $I$ is the identity matrix.
    \end{corollary}

    \ti{Proof.}
    To see that the $I - \gamma P^\pi$ is invertible ($\iff \ker{I - \gamma P^\pi} = \{0\}$), It is enough to show that $\forall x \neq 0, x \in \mbb{R}^{\abs{\mc{S}}\abs{\mc{A}}}, \norm{(I - \gamma P^\pi) x}_\infty > 0$.

    \[
    \begin{aligned}
        \norm{(I - \gamma P^\pi)x}_\infty &= \norm{x - \gamma P^\pi x}_\infty \\ &\geq \norm{x}_\infty - \gamma \norm{P^\pi x}_\infty &(\text{reverse triangle inequality})\\
        &\geq \norm{x}_\infty - \gamma \norm{x}_\infty &(\text{each element of }P^\pi x\text{ is an average of }x ) \\
        &=(1-\gamma) \norm{x}_\infty > 0 &(\gamma < 1, x \neq 0)
    \end{aligned}
    \]
    This implies $(I - \gamma P^\pi)x = 0 \implies x = 0$ and matrix $I - \gamma P^\pi$ is full rank
\end{frame}

\begin{frame}{MDP}
    \begin{lemma}
        \[
        \begin{gathered}
            \left[(1 - \gamma)(I - \gamma P^\pi)^{-1}\right]_{(s,a), (s^\prime, a^\prime)} \\
            = (1 - \gamma) \sum_{t=0}^\infty \gamma^t Pr(S_t=s^\prime, A_t = a^\prime | S_0 =s, A_0 = a; \pi)
        \end{gathered}
        \]
    \end{lemma}

    \ti{proof.}
    
\end{frame}


\end{document}