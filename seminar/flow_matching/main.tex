\documentclass[11pt]{beamer}
\usefonttheme[onlymath]{serif}

\setbeamersize{text margin left=1.5em}
\setbeamersize{text margin right=1.5em}

\setbeamertemplate{frametitle}{%
  \vskip1ex
  \usebeamerfont{frametitle}%
  \insertframetitle\par
  \vskip1ex
  \hrule
}

\setbeamertemplate{blocks}[rounded][shadow=true]

\setbeamertemplate{itemize item}{\usebeamerfont{itemize item}\textbullet}
\setbeamertemplate{itemize subitem}{\usebeamerfont{itemize subitem}\textbullet}
\setbeamertemplate{itemize subsubitem}{\usebeamerfont{itemize subsubitem}\textbullet}

\makeatletter
\geometry{%
  papersize={\fpeval{\beamer@paperwidth*1.5}pt,\fpeval{\beamer@paperheight*1.5}pt},
  hmargin=\fpeval{0.5 * 1.5}cm,% 1cm
  vmargin=0cm,%
  head=\fpeval{0.5*1.5}cm,% 0.5cm
  headsep=0pt,%
  foot=\fpeval{0.5*1.5}cm% 0.5cm
}
\makeatother

\usepackage{graphicx}
\usepackage{amsmath}
\usepackage{amssymb}
\usepackage{mathtools}

\usepackage[
  backend=bibtex,
  style=authoryear,   % or numeric
  citestyle=authoryear
]{biblatex}
\addbibresource{../../references.bib}

% Your custom commands (kept as is)
\newcommand{\mrm}[1]{\mathrm{#1}}
\newcommand{\mbb}[1]{\mathbb{#1}}
\newcommand{\mb}[1]{\mathbf{#1}}
\newcommand{\mc}[1]{\mathcal{#1}}
\newcommand{\tb}[1]{\textbf{#1}}

\title{Flow Matching for Generative Modeling}
\author{Gwanwoo Choi}
\institute{MLIC}
\date{} % To use the current date

\begin{document}

\begin{frame}
    \titlepage
\end{frame}

\begin{frame}{VAE, Diffusion, FlowMatching}

  \begin{figure}
    \includegraphics[width=0.6\textwidth]{VAELatent.png}
  \end{figure}

  \begin{block}{VAE Loss (\cite{kingmaAutoEncodingVariationalBayes2022})}
    $$
    \begin{gathered}
      \mc{L}(\theta, \phi) \triangleq \mbb{E}_{x \sim \mc{D}} \left[D_{KL}(q_\phi(z|x), p_{\theta}(z)) - \mbb{E}_{z \sim q_\phi(\cdot|x)}\left[\log p_{\theta}(x|z)\right]\right]
    \end{gathered}
    $$

    \begin{itemize}
      \item we want distribution of $z$ directly matches the complex and untractable data distribution of $x$
      \item $z$ directly matches to $x$ at once
      \item Ideal distribution of $z$ should be tractable (e.g. $\mc{N}(0, I)$)
    \end{itemize}
  \end{block}
\end{frame}

\begin{frame}{VAE, Diffusion, FlowMatching}
  \begin{figure}
    \includegraphics[width=0.75\textwidth]{DDPMExample.png}
  \end{figure}

  \begin{block}{Diffusion Process (\cite{hoDenoisingDiffusionProbabilistic2020a})}
    \begin{itemize}
      \item In diffusion process, latent space becomes $x_T$ and data space becomes $x_0$ ($T$ is an arbitrary timestep)
      \item Rather than going directly from $x_T$ to $x_0$ at once, split the path in separate $T$ timesteps.
      \item For all $t$, each $x_t$ space can be obtained recursively by sampling from distribution
      $q(x_t|x_{t-1}) = \mc{N}(\sqrt{1-\beta_t}x_{t-1}, \beta_t I)$ ($\beta_t$ is an arbitrary constant)
      \item Simiarly with VAE, we can also construct the distribution $p_{\theta}(x_{t-1}|x_t)$
      \item This distribution is intractable, so $p_\theta$ distribution is parameterized by $\theta$
    \end{itemize}
  \end{block}
\end{frame}

\begin{frame}{VAE, Diffusion, FlowMatching}
  \begin{figure}
    \includegraphics[width=0.75\textwidth]{DDPMExample.png}
  \end{figure}

  \begin{block}{Diffusion Loss}
    $$
      \mbb{E}_{(x_T, x_{T-1}, \dots, x_0) \sim q} \left[ \underbrace{D_{\text{KL}}(q(x_T|x_0)|p(x_T))}_{L_T} + \sum_{t>1} \underbrace{D_{\text{KL}}(q(x_{t-1}|x_t,x_0)|p_\theta (x_{t-1}|x_t))}_{L_{t-1}} - \underbrace{\log p_{\theta} (x_0|x_1)}_{L_0} \right]
    $$
  \end{block}
\end{frame}

\printbibliography

\end{document}

