\documentclass{article}
\usepackage{amsmath}
\usepackage{booktabs}
\usepackage{xcolor}
\usepackage{array}

% 1. 중앙 정렬 컬럼 정의
\newcolumntype{M}[1]{>{\centering\arraybackslash}m{#1}}

% 2. 색상 정의
\definecolor{boxMint}{RGB}{197, 247, 234}
\definecolor{lineOrange}{RGB}{240, 156, 78}
\definecolor{boxPurple}{RGB}{135, 111, 239}

% 3. [핵심] "투명 박스" 만들기 명령어 정의
% - 1.8em 너비의 투명 박스를 만들고, 그 안에서 도형을 중앙(c) 정렬합니다.
% - 이제 모든 도형은 1.8em 만큼의 공간을 똑같이 차지합니다.
\newcommand{\iconBox}[1]{\makebox[1.8em][c]{#1}}

\begin{document}

\begin{table}[h]
\centering
\renewcommand{\arraystretch}{1.8}

\begin{tabular}{ M{1cm} M{3cm} M{3cm} } 
\toprule[1.5pt]

\textbf{Idx.} & \textbf{Method} & \textbf{Hyperparams} \\ 
\midrule

% --- (a) ---
\textbf{(a)} & 
% \iconBox로 감싸서 너비 고정
\iconBox{\textcolor{boxMint}{\rule[-0.3em]{1.2em}{1.2em}}} TCE(NN) & 
$\begin{gathered}
    \lambda_{\text{cov}} \in (0,1] \\
    \lambda_{\text{mix}} \in (0,1]
\end{gathered}$ \\ 
\cmidrule(lr){1-3}

% --- (b) ---
\textbf{(b)} & 
\iconBox{\textcolor{lineOrange}{\rule[0.4em]{1.5em}{2.5pt}}} TCE(OG) & 
$\begin{gathered}
    \lambda_{\text{cov}} \in (0,1] \\
    \lambda_{\text{mix}} = 0
\end{gathered}$ \\ 
\cmidrule(lr){1-3}

% --- (c) ---
\textbf{(c)} & 
% 도형이 작아도 iconBox 덕분에 공간은 (a)와 똑같이 차지함 -> 글자 위치 고정됨
\iconBox{\textcolor{boxPurple}{\rule[-0.2em]{0.8em}{0.8em}}} Simple Aug. & 
$\begin{gathered}
    \lambda_{\text{cov}} = 0 \\
    \lambda_{\text{mix}} = 0
\end{gathered}$ \\ 

\bottomrule[1.5pt]
\end{tabular}
\end{table}

\end{document}