\documentclass[
	10pt, % Set the default font size, options include: 8pt, 9pt, 10pt, 11pt, 12pt, 14pt, 17pt, 20pt
	%t, % Uncomment to vertically align all slide content to the top of the slide, rather than the default centered
	%aspectratio=169, % Uncomment to set the aspect ratio to a 16:9 ratio which matches the aspect ratio of 1080p and 4K screens and projectors
]{article}
\usepackage{booktabs} % Allows the use of \toprule, \midrule and \bottomrule for better rules in tables
\usepackage{amsmath} % Automatically numbers and align the equations
\usepackage{amsfonts} % To use mathematical fonts like mathbb
\usepackage[hidelinks]{hyperref} % hide red rectangle on reference. hyperref is used packaged as default.
\usepackage{cleveref} % clearly reference equation in text
\usepackage{braket} % braket
\usepackage{lmodern} %font for equation
\usepackage[default]{opensans} % Use the Open Sans font for sans serif 
\usepackage{bm} %access bold symbols in math modes



\def\tcr{\textcolor{red}}
\def\tcb{\textcolor{blue}}
\def\n{\newline}

\newcommand{\mrm}[1]{\mathrm{#1}}
\newcommand{\mbb}[1]{\mathbb{#1}}
\newcommand{\mc}[1]{\mathcal{#1}}
\newcommand{\tb}[1]{\textbf{#1}}
\newcommand{\ti}[1]{\textit{#1}}

\numberwithin{equation}{subsection} % command in amsmath package
% define argmax argmin (https://tex.stackexchange.com/questions/5223/command-for-argmin-or-argmax)
% \DeclareMathOperator*{\argmax}{arg\,max} % not good
% \newcommand{\argmax}{\mathop{\mathrm{arg\,max}}\limits}
% \newcommand{\argmin}{\mathop{\mathrm{arg\,min}}\limits}
\newcommand{\argmax}{\mathop{\mathrm{argmax}}\limits}
\newcommand{\argmin}{\mathop{\mathrm{argmin}}\limits}



% \setlength{\parindent}{15pt} % set indent length
\title{Reinforcement Learning basic study}
\author{Gwanwoo Choi}

\begin{document}
\maketitle
\newpage
\tableofcontents
\newpage
\section{Markov Process}
In probability theory and statistics, a markov process is a stochastic process describing a sequence of possible events in which the probability of each event depends only on the state attained in the previous event.
If stochastic process satisfies \cref{eq:eq1}, then that is \tb{markov process} and state $S_t$ is \tb{markov}.
\begin{equation} \label{eq:eq1}
    \mbb{P}[S_{t+1}|S_t] = \mbb{P}[S_{t+1}|S_t,S_{t-1},\dots,S_{1}]
\end{equation}
For a markov state $s$ and successor state $s'$, the state transition probability is defined by \cref{eq:eq2}.
\begin{equation} \label{eq:eq2}
    \mc{P}_{ss'} = \mbb{P}[S_{t+1} = s' | S_t = s]
\end{equation}
Let $\mc{S}$ be a set of all states and $\mc{P}_{ss'}$ be a set of all state-transition probability.
 \ti{markov process} is uniquely determined by a tuple $\braket{\mc{S}, \mc{P}}$. 


\subsection{Markov Reward Process}


$R_t$ is called \tb{reward} for corresponding state $S_t$. \tb{Return} $G_t$ is defined by \cref{eq:eq3}.   
\begin{equation} \label{eq:eq3}
    G_t = R_{t+1} + \gamma R_{t+2} + \gamma^2 R_{t+3} + \dots = \sum^{\infty}_{k=0}\gamma^k R_{t+k+1}
\end{equation}
where $\gamma \ \in [0,1] $ is called \tb{discount factor}.
$\mc{R}_s$ is defined by \cref{eq:eq4}
\begin{equation} \label{eq:eq4}
    \mc{R}_s = \mbb{E}[R_{t+1}|S_t=s]
\end{equation}
And let denote $\mc{R}$ as set of all $\mc{R}_s$ for all $s \in \mc{S}$. Then \tb{markov reward process} is uniquely determined by a tuple $\braket{\mc{S}, \mc{P}, \mc{R}, \gamma}$.

The \tb{value funtion} $v(s)$ represents long-term value of state s. \ti{value function} is defined by \cref{eq:eq5}.
\begin{equation} \label{eq:eq5}
    v(s) = \mbb{E}[G_t|S_t=s]
\end{equation}


\subsection{Markov Decision Process}
$A_t$ is called \tb{action}. For each time step $t$, action $A_t$ is elements of the set $\mc{A}$ ($\forall t, A_t \in \mc{A}(s) \text{ where } s \in \mc{S}$). This action set can be both finite and infinite set. Unlike transition probability in \ti{markov process}, \tb{transition probability} is newly deefined in markov decision process as \cref{eq:eq8}. Also \tb{reward function} is newly defined as \cref{eq:eq9}.
\begin{equation} \label{eq:eq8}
    \mc{P}_{ss'}^a = \mbb{P}[S_{t+1}=s'|S_t=s, A_t=a]
\end{equation}
\begin{equation} \label{eq:eq9}
    \mc{R}_{s}^a = \mbb{E}[R_{t+1}|s_{t}=s,A_t=a]
\end{equation}
similarly with markov process and markov reward process, markov decision process is uniquely determined by a tuple $\braket{\mc{S}, \mc{A}, \mc{P}, \mc{R}, \gamma}$

Now we can define the probability distribution over actions given states, named \tb{policy} $\pi$ as \cref{eq:eq10}.
\begin{equation} \label{eq:eq10}
    \pi(a|s) = \mbb{P}[A_t=a|S_t=s]
\end{equation}
A policy fully defines the behaviour of agent.
\n

Gieven an MDP $\mc{M}=\braket{\mc{S, A, P, R}, \gamma}$ and a policy $\pi$, $\braket{\mc{S}, \mc{P}^\pi}$ is \ti{MP} and $\braket{\mc{S},\mc{P}^\pi, \mc{R}^\pi, \gamma}$ is \ti{MRP}. $\mc{P}^\pi$ is set of all $\mc{P}^\pi_{s,s'}$ which is defined by \cref{eq:eq11}.
\begin{equation} \label{eq:eq11}
    \mc{P}^\pi_{s,s'} = \sum_{a \in \mc{A}}\pi(a|s)\mc{P}^a_{ss'}
\end{equation}
And $R^\pi$ is set of all $R^pi_s$ which is defined by \cref{eq:eq12}
\begin{equation} \label{eq:eq12}
    \mc{R}^\pi_s = \sum_{a \in \mc{A}}\pi(a|s)\mc{R}^a_s
\end{equation}
\n

In MDP, we can define \tb{value function} as new manner by \cref{eq:eq13}.
\begin{equation} \label{eq:eq13}
    v_\pi(s) = \mbb{E}_\pi[G_t|S_t=s]
\end{equation}
Also, \tb{action-value function} $q_\pi(s,a)$, which is the expected return starting from state s, taking action a, and then following policy $\pi$, is defined by \cref{eq:eq14}.
\begin{equation} \label{eq:eq14}
    q_\pi(s,a) = \mbb{E}_\pi[G_t|S_t=s, A_t=a]
\end{equation}
\n

\section{Bellman Equation}
\subsection{Bellman Equation for MRP and MDP}
The \tb{bellman equation}, which tells returns can be decomposed into immediate reward and discounted (action) value of successor state, is described as below.
\cref{eq:eq6} represents \ti{Bellman equation for MRPs}.
\begin{equation} \label{eq:eq6}
    \begin{aligned}
        v(s) &= \mbb{E}[G_{t+1}|S_t = s] \\
        &= \mbb{E}[R_{t+1} + \gamma v(S_{t+1})|S_t = s] \\
        &= \mc{R}_s + \gamma \sum_{s' \in \mc{S}} \mc{P}_{ss'}v(s')
    \end{aligned}
\end{equation}
value function can be calculated by utilizing $\gamma, \mc{P} \text{ and } \mc{R}$. 
In vector formulation, value funtion can be obtained by \cref{eq:eq7}.
\begin{equation} \label{eq:eq7}
    v = (I - \gamma \mc{P})^{-1}\mc{R}
\end{equation}

\cref{eq:eq15} represents \ti{Bellman equation} for \ti{MDP}.
\begin{equation} \label{eq:eq15}
    \begin{aligned} 
        v_\pi(s)&=\mbb{E}_\pi[R_{t+1}+\gamma v_\pi(S_{t+1})|s_t=s]\\
        &=\sum_{a \in \mc{A}}\pi(a|s)\left(\mc{R}^a_{s}+\gamma\sum_{s' \in \mc{S}}\mc{P}^a_{ss'}v_\pi(s')\right)\\
        q_\pi(s,a)&=\mbb{E}_\pi[R_{t+1}+\gamma q_\pi(S_{t+1},A_{t+1})|S_t=s,A_t=a]\\
        &=\mc{R}^a_s+\gamma \sum_{s' \in \mc{S}}\mc{P}_{ss'}^a \sum_{a' \in \mc{A}} \pi(a'|s')q_\pi(s',a')
    \end{aligned}
\end{equation}
this value function and action value function can be expressed mixedly.
\begin{equation} \label{eq:eq16}
    \begin{gathered}
        v_\pi(s) = \sum_{a \in \mc{A}}\pi(a|s)q_\pi(s,a)\\
        q_\pi(s,a) = \mc{R}^a_s + \gamma\sum_{s' \in \mc{S}}\mc{P}^a_{ss'}v_\pi(s')
    \end{gathered}
\end{equation}

Similarly with \cref{eq:eq7}, bellman equation related with a specific policy can be expressed by \cref{eq:eq19}
\begin{equation} \label{eq:eq19}
    v_\pi = \mc{R}^\pi + \gamma \mc{P}^\pi v_\pi\\
\end{equation}
And we can get direct solution with process such as \cref{eq:eq20}
\begin{equation} \label{eq:eq20}
    v_\pi = (I - \gamma \mc{P}^\pi)^{-1} \mc{R}^\pi
\end{equation}

The time complexity is $\mc{O}(n^3)$ for calculating direct solution where $n$ is $|\mc{S}|$.
So direct solution is suitable for small MRPs (where $n$ is small).
There exists indirect iterative methods such as \ti{DP}, \ti{MC} and \ti{TD} which is suitable for calculating solution larger MRPs.

\subsection{Optimal Value Functions}
Value functions define a partial ordering over policies. A policy $\pi$ is said to be better than or equal to a policy $\pi'$ (i.e. $\pi \geq \pi'$) if and only if $v_\pi(s) \geq v_{\pi'}(s)$.
\tb{Optimal state-value function} for all state $s$ is defined by \cref{eq:eq17}.
\begin{equation} \label{eq:eq17}
    v_*(s) = \max_{\pi} v_\pi(s)
\end{equation}
\tb{Optimal action-value function} for all state and action pair $(s, a)$ is defined bt \cref{eq:eq18}
\begin{equation} \label{eq:eq18}
    % \begin{aligned}
    q_*(s,a) = \max_\pi q_\pi(s,a)
    % &=\mbb{E}[R_{t+1} + \gamma v_*(S_{t+1})|S_t=s,A_t=a]
    % \end{aligned}
\end{equation}

\subsection{Bellman Optimality Equation}
The \tb{bellman optimality equation} are special consistency conditions that the optimal value functions must satisfy and that can, in principle, be solved for the optimal value functions, from which an optimal policy can be determined with relative ease. Bellman optimality equation for state value function can be expressed like \cref{eq:eq21}
\begin{equation} \label{eq:eq21}
    \begin{aligned}
        v_*(s) &= \max_{a}q_{\pi_*}(s,a)\\
        &=\max_a \mc{R}^a_{s} + \gamma \sum_{s' \in \mc{S}}\mc{P}^a_{ss'}v_*(s')
    \end{aligned}
\end{equation}
And bellman optimality equation for action value function can be expressed like \label{eq:eq22}
\begin{equation} \label{eq:eq22}
    \begin{aligned}
        q_*(s,a) &= \mc{R}^a_{s} + \gamma \sum_{s' \in \mc{S}} \mc{P}^a_{ss'}v_*(s') \\
        &= \mc{R}^a_{s} + \gamma \sum_{s' \in \mc{S}} \mc{P}^a_{ss'} \max_{a'} q_*(s',a')
    \end{aligned}
\end{equation}



Whereas the \ti{optimal value functions} for states and state-action pairs are \ti{unique} for states and state-action pairs are unique, there can be \ti{many optimal policies}.

\subsection{Optimal Policy}
There is always at least one policy that is better than or equal to all other policies. This is an \tb{optimal policy}. All optimal policies achieve the optimal state value function, $\bm{v_{\pi_*}(s) = v_*(s)}$ for all state $s$ and the optimal action value function $\bm{q_{\pi_*}(s,a) = q_*(s,a)}$ for all state and action pair $(s,a)$. In other words, \ti{any policy that is greedy} with respect to the optimal value function $v_*$ is an \ti{optimal policy}.

An \ti{optimal policy} $\pi_*$ can be found by maximising over $q_*(s,a)$,
\begin{equation}
    \pi_*(a|s) = 
    \begin{cases}
        1 & \text{if } a = \argmax_{a \in \mc{A}}q_*(s,a) \\
        0 & \ti{otherwise}
    \end{cases}
\end{equation}
if we know $q_*(s,a)$ we immediate have the optimal policy.
\newpage



\section{Dynamic Programming}
In this chapter, we only consider about deterministic policies, although all properties and equations can also be able to applied for stochastic policy $\pi(a|s)$.
\subsection{Policy Evaluation}
Suppose there exists a MDP $\braket{\mc{S,A,P,R}, \gamma}$ and a given policy $\pi$. How we can get value function for specific policy? Throught \tb{Policy evaluation}, also called as \tb{prediction problem}, we can evaluate policy.

If we know the environment's dynamics, then we can consider \cref{eq:eq6} as a system of $|\mc{S}|$ linear equations of $|S|$ unknowns. In other words, by solving that $|\mc{S}|$ number of linear equations we can get value function for a policy $\pi$. However solving such tremendous linear system involves tedious calculations and can lead to lack of computation resources. For our purpose, iterative solution methods are most suitable. \tb{Iterative update process (Jacobi's method)} for state value function can be expressed by \cref{eq:eq23}.
\begin{equation} \label{eq:eq23}
    \begin{aligned}
        v_{k+1}(s) &= \mbb{E}_\pi [R_{t+1} + \gamma v_k(S_{t+1}) | S_t = s] \\
        &= \sum_{a \in \mc{A}(s)} \pi(a|s)\left(\mc{R}^a_s + \gamma\sum_{s' \in \mc{S}}\mc{P}^a_{ss'}v_{k}(s') \right)
    \end{aligned}
\end{equation}

Note that not only Jacobi's method, but also Gauss-Seidal, SOR(Successive over relaxation) and other iterative methods can be used for policy evaluation (prediction problem).

\subsection{Policy Improvement}
Suppose we have determined the value function $v_\pi$ for a policy for an arbitrary policy $\pi$. Assume another policy $\pi'$ satisfies inequation $q_\pi(s, \pi'(s)) \geq v_\pi(s)$, for all $s \in \mc{S}$. Then following inequation \cref{eq:eq24} is satisfied.
\begin{equation} \label{eq:eq24}
    \begin{aligned}
        v_\pi(s) &\leq q_\pi(s, \pi'(s))\\
        &=\mbb{E}[R_{t+1} + \gamma v_\pi(S_{t+1}) | S_t=s, A_t=\pi'(s)]\\
        &=\mbb{E}_{\pi'}[R_{t+1} + \gamma v_\pi (S_{t+1}) | S_{t}=s]\\
        &\leq \mbb{E}_\pi'[R_{t+1} + \gamma q_\pi(S_{t+1}, \pi'(S_{t+1}))|S_t=s]\\
        &= \mbb{E}_{\pi'}[R_{t+1} + \gamma\mbb{E}_{\pi'}[R_{t+2} + \gamma v_\pi(S_{t+1})|S_{t+1}]|S_t=s]\\
        &= \mbb{E}_{\pi'}[R_{t+1}+\gamma R_{t+2} + \gamma^2 v_\pi (S_{t+2}) |S_t=s] \quad \text{(by law of iterated expectation)}\\
        &\vdots\\
        &\leq \mbb{E}_{\pi'}[G_{t+1}|S_t=s]\\
        &= v_{\pi'}(s)
    \end{aligned}
\end{equation}

And for every policy $\pi$, we can always find better or equal policy $\pi'$. Once we get value function $v_\pi$ for $\pi$, $\pi'$ defined by \cref{eq:eq25} is always better than $\pi$.
\begin{equation} \label{eq:eq25}
    \pi'(s) = \argmax_a q_\pi(s,a)
\end{equation}
This implies new greedy policy for value function of original policy is always better than original policy and The process of deriving a new policy using the aforementioned method is called \tb{policy improvement}.

If new policy $\pi'$ derived from original policy $\pi$ satisfies bellman optimality equation (\cref{eq:eq26}), then both $\pi'$ and $\pi$ are optimal policy.
\begin{equation} \label{eq:eq26}
    v_{\pi'}(s) = \max_a \mbb{E}[R_{t+1} + \gamma v_{\pi'}(S_{t+1}) | S_t=s, A_t=a]
\end{equation}

\subsection{Policy Iteration}
We can recursively update the value function and policy in sequence manner like \cref{eq:eq27}.
\begin{equation} \label{eq:eq27}
    \pi_0 \xrightarrow{\mrm{E}} v_{\pi_0} \xrightarrow{\mrm{I}} \pi_1 \xrightarrow{\mrm{E}} v_{\pi_1} \xrightarrow{\mrm{I}} \dots \xrightarrow{\mrm{E}} \pi_*
\end{equation}
Terminate when newly updated policy $\pi_{k+1}$ is eqaul to $\pi_k$ and this policy $\pi_{k+1}$ becomes optimal policy $\pi_*$. This process used to obtain optimal policy $\pi_*$ is called \tb{policy iteration}.

\subsection{Value Iteration}
Waiting to complete policy evaluation at each step of policy iteration can lead to wasted time. In fact, the policy evaluation step in policy iteration can be truncated in several ways without losing the convergence guarantees of policy iteration. \tb{Value iteration} truncates the policy evaluation step to just \ti{one step}, ensuring fast convergence. The update equation for value iteration can be written as a simple one-line equation (\cref{eq:eq28}) that combines policy improvement and the one-step truncated policy evaluation step.
\begin{equation} \label{eq:eq28}
    v_{k+1}(s) = \max_a \mbb{E} [R_{t+1} + \gamma v_k(S_{t+1}) | S_t=s, A_t=a]
\end{equation}

\end{document}
