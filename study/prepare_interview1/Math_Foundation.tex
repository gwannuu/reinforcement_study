\documentclass[8pt]{beamer}
\usefonttheme[onlymath]{serif}


\setbeamerfont{frametitlefont}{size=\Large}
\setbeamerfont{framesubtitlefont}{size=\large}

\setbeamertemplate{frametitle}{%
  \vskip1ex
  \usebeamerfont{titlebig}%
  \insertsectionhead
  \usebeamerfont{framesubtitlefont}
  \!:\insertsubsectionhead
  \par        %  ← 원하는 대로 변경 가능
  \vskip1ex
  \hrule                             % 밑줄(선택)
}

% 테마 선택 (선택 사항)
% \usetheme{Madrid} % 기본 테마, 다른 테마 사용 가능
% \font{serif}
\usepackage{amsfonts}
\usepackage{amssymb}
\usepackage[T1]{fontenc} % To use combination of textbf, textit
\usepackage[dvipsnames]{xcolor}   % can use more variant colors

% \setcounter{MaxMatrixCols}{20}

% (필요한 패키지들)
% \usepackage{amsthm}
\setbeamertemplate{theorems}[numbered]  % 정리, 정의 등에 번호를 달아줌

% \theoremstyle{plain} % insert bellow all blocks you want in italic
% \newtheorem{theorem}{Theorem}[section] % to number according to section
% 
% \theoremstyle{definition} % insert bellow all blocks you want in normal text
% \newtheorem{definition}{Definition}[section] % to number according to section
% \newtheorem*{idea}{Proof idea} % no numbered block

\newtheorem{proposition}[theorem]{Proposition}

\usepackage{tcolorbox}

% 필요할 경우 패키지 추가
\usepackage{graphicx} % 이미지 삽입을 위한 패키지
\usepackage{amsmath}   % 수식 사용
\usepackage{hyperref}  % 하이퍼링크 추가
\usepackage{cleveref}
\usepackage{multicol}  % 여러 열 나누기
\usepackage{ulem} % 취소선 및줄 나누기



\newcommand{\mrm}[1]{\mathrm{#1}}
\newcommand{\mbb}[1]{\mathbb{#1}}
\newcommand{\mb}[1]{\mathbf{#1}}
\newcommand{\mc}[1]{\mathcal{#1}}
\newcommand{\tb}[1]{\textbf{#1}}
\newcommand{\ti}[1]{\textit{#1}}
\newcommand{\mypois}[1]{\operatorname{Pois}(#1)}

\newcommand{\myber}[1]{\operatorname{Bern}\!\left(#1\right)}
\newcommand{\mybin}[2]{\operatorname{Bin}\!\left(#1,#2\right)}
\newcommand{\mytoinf}[1]{#1 \rightarrow \infty}
\newcommand{\myexp}[1]{\exp{\left(#1\right)}}
\newcommand{\myunif}[2]{\operatorname{Unif}\!\left(#1, #2\right)}
\newcommand{\mygeom}[1]{\operatorname{Geom}\!\left(#1\right)}
\newcommand{\myexpo}[1]{\operatorname{Expo}\!\left(#1\right)}
\newcommand{\abs}[1]{\left\lvert #1 \right\rvert}
\newcommand{\expec}[1]{\operatorname{E}\left[ #1 \right]}
\newcommand{\myvar}[1]{\operatorname{Var}\left[#1\right]}
\newcommand{\myskew}[1]{\operatorname{Skew}\!\left[#1\right]}
\newcommand{\mykurt}[1]{\operatorname{Kurt}\!\left[#1\right]}
\newcommand{\mywei}[2]{\operatorname{Wei}\!\left(#1, #2\right)}
\newcommand{\Span}[1]{\operatorname{Span}\!\left(#1\right)}
\newcommand{\argmax}[1]{\operatorname{arg max}_{#1}}
\newcommand{\argmin}[1]{\operatorname{arg min}_{#1}}
\newcommand{\nll}[1]{\operatorname{NLL}\!\left(#1\right)}
\newcommand{\rss}[1]{\operatorname{RSS}\!\left(#1\right)}
\newcommand{\tr}[1]{\operatorname{tr}\!\left(#1\right)}

% 발표 제목, 저자, 날짜 설정
\title{Math Foundation}
\author{Gwanwoo Choi}
% \date{}

\begin{document}
% 표지 슬라이드
\begin{frame}
    \titlepage
\end{frame}

\begingroup
    \setbeamertemplate{frametitle}{%
        \vskip1ex
        \usebeamerfont{frametitle}%
        \insertframetitle\par
        \vskip1ex
        \hrule
    }%
    \begin{frame}
        \frametitle{Table of Content}
        \tableofcontents
    \end{frame}
\endgroup

\section{Linear Algebra}
\subsection{Norms of a vector and matrix}

\begin{frame}{.}
    \begin{definition}
        In vector space $\mbb{R}^d$, A norm of vector $\abs{x}$ is a measure of how big a vector is. Norm is any function $f: \mbb{R}^d \rightarrow \mbb{R}$ that satisfies under condition.
        \begin{itemize}
            \item $\abs{x} \geq 0, \forall x\in \mbb{R}^d$
            \item $\abs{x} = 0 \iff x = 0$
            \item $\abs{cx} = c\abs{x}, \forall c \in \mbb{R}$
            \item $\abs{x} + \abs{y} \geq \abs{x+y}, \forall x,y \in \mbb{R}^d$
        \end{itemize}
    \end{definition}

    Example of several norms satisfying these conditions.

    \tb{$p$-norm}: $\abs{x}_p = \sqrt[p]{\sum_{i=1}^d \abs{x_i}^p}$, for $p \geq 1$

    \tb{max-norm ($\infty$-norm)}: $\abs{x}_\infty = \max_i \abs{x_i}$

    $0$-norm: $\abs{x}_0 = \sum_{i=1}^d \mb{1}(\abs{x_i} > 0 )$
\end{frame}

\begin{frame}{.}
    If we think matrix $A \in \mbb{R}^{m \times n}$ as linear function $f: \mbb{R}^n \rightarrow \mbb{R}^m$, $x \mapsto f(x) = Ax$, we can define \tb{induced norm} of that function $f$ by 

    \[\abs{A} = \max_{x\neq 0}\frac{\abs{Ax}_p}{\abs{x}_p} = \max_{\abs{x} = 1} \abs{Ax}_p\]

    Especially, $p=2$,
    \[
    \abs{A}_2 = \max_{\abs{x} = 1} \abs{Ax}_2\ = \max_i \sigma_i
    \] where $\sigma_i$ is the $i$-th singular value of $A$.

    Also there exists called \tb{nuclear norm} or \tb{trace norm}, is
    \[
    \abs{A}_\ast = \tr{\sqrt{A^\top A}} = \sum_i \sigma_i = \sum_i \abs{\sigma_i}
    \]

    More generalized version of trace norm, \tb{Schatten $p$-norm} is defined by
    \[
    \abs{A}_p = \sum_i \left(\abs{\sigma_i}^p\right)^{1/p}
    \]
\end{frame}

\end{document}