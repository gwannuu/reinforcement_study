\documentclass{beamer}
\usefonttheme[onlymath]{serif}
% 테마 선택 (선택 사항)
\usetheme{Madrid} % 기본 테마, 다른 테마 사용 가능
% \font{serif}
\usepackage{amsfonts}
\usepackage{amssymb}

% \setcounter{MaxMatrixCols}{20}

% (필요한 패키지들)
% \usepackage{amsthm}
\setbeamertemplate{theorems}[numbered]  % 정리, 정의 등에 번호를 달아줌

% \theoremstyle{plain} % insert bellow all blocks you want in italic
% \newtheorem{theorem}{Theorem}[section] % to number according to section
% 
% \theoremstyle{definition} % insert bellow all blocks you want in normal text
% \newtheorem{definition}{Definition}[section] % to number according to section
% \newtheorem*{idea}{Proof idea} % no numbered block
\usepackage{tcolorbox}

% 필요할 경우 패키지 추가
\usepackage{graphicx} % 이미지 삽입을 위한 패키지
\usepackage{amsmath}   % 수식 사용
\usepackage{hyperref}  % 하이퍼링크 추가
\usepackage{cleveref}
\usepackage{multicol}  % 여러 열 나누기


\newcommand{\mrm}[1]{\mathrm{#1}}
\newcommand{\mbb}[1]{\mathbb{#1}}
\newcommand{\mb}[1]{\mathbf{#1}}
\newcommand{\mc}[1]{\mathcal{#1}}
\newcommand{\tb}[1]{\textbf{#1}}
\newcommand{\ti}[1]{\textit{#1}}

% 발표 제목, 저자, 날짜 설정
\title{Probability}
\author{Gwanwoo Choi}
% \date{}

\begin{document}

% 표지 슬라이드
\begin{frame}
    \titlepage
\end{frame}

% 목차 슬라이드
\begin{frame}{contents}
    \tableofcontents
\end{frame}

\section{Expectation}

\begin{frame}
    \frametitle{Table of Contents}
    \tableofcontents[currentsection]
\end{frame}


\begin{frame}{Expectation}
    \begin{definition}[Expectation of a discrete r.v.]
        The \tb{expected value} (also called the \tb{expectation} or \tb{mean}) of a discrete r.v. $X$ whose distinct possible values are $x_1, x_2, \dots$ is defined by 
        \[
            E(X) = \sum_x x P(X=x)
        \]
        or infinite form
        \[
            E(X) = \sum^\infty_{j=1} x_j P(X=x_j)
        \]

    \end{definition}
\end{frame}

\begin{frame}{Expectation}
    \begin{example}[Bernoulli expectation]
        Let $X \sim \text{Bern}(p)$ and $q = 1-p$.
        \[
            E(X) = 1p +0q = p
        \]
    \end{example}
\end{frame}

\begin{frame}{Expectation}
    \begin{theorem}[Linearity of expectation]
        For any r.v.s $X$, $Y$ and any constant $c$, 
        \[
        \begin{gathered}
            E(X+Y)=E(X) + E(Y)\\
            E(cX) = cE(X)
        \end{gathered}
        \]
    \end{theorem}
\end{frame}

\begin{frame}{Expectation}
    \begin{example}[Binomial expectation]
        For $X \sim \text{Bin}(n,p)$
        \[
            \begin{aligned}
                E(X) &= \sum_{k=0}^n k P(X=k) = \sum_{k=0}^n k \binom{n}{k} p^k q^{n-k}\\
                &= n \sum^n_{k=0} \binom{n-1}{k-1} p^k q^{n-k} \quad (\because k \binom{n}{k} = n \binom{n-1}{k-1})\\
                &= np \sum^n_{k=1} \binom{n-1}{k-1} p^{k-1} q^{n-k} = np \sum_{j=0}^{n-1} \binom{n-1}{j}p^j q^{n-1-j} \\
                &= np \quad (\because \binom{n-1}{j}p^j q^{n-1-j} = (p + q)^{n-1} = 1)
            \end{aligned}
        \]
        For $X = I_1 + \dots + I_n, I_k \sim Berp(p)$, $E(X) = E(I_1) + \cdots + E(I_n) = np$
    \end{example}
\end{frame}


\begin{frame}{Expectation}
    \begin{definition}[Geometric distribution]
        Consider a sequence of independent Bernoulli trials, each with the same success probability $p \in (0,1)$, with trials repeated until success. Let $X$ be the number of trials before the first successful trial. Then $X$ has a \tb{geometric distribution} with parameter $p$. we denote this by $X \sim \text{Geom}(p)$.\newline
    \end{definition}

    Remine TD$(\lambda)$
    \[
        G^\lambda_t = (1-\lambda) \sum^\infty_{n=1} \lambda^{n-1} G_{t:t+n}
    \]
\end{frame}

\begin{frame}{Expectation}
    \begin{theorem}[Geometric PMF]
        If $X\sim\text{Geom}(p)$, then the PMF of $X$ is 
        \[
            P(X=k) = q^k p
        \]
        for $k = 0,1,2,\dots, $ wehre $Q = 1-p$
    \end{theorem}
    Note that summing a geometric series, we have
    \[
        \sum^\infty_{k=0} q^k p = p \sum^\infty_{k=0} q^k = p \cdot \frac{1}{1-q} = 1
    \]
    (Sum of probabilities should be $1$)
\end{frame}

\begin{frame}{Expectation}
    \begin{example}[Geometric expectation] \label{ex:geometric_expectation}
        Let $X \sim \text{Geom}(p)$. By definition,
        \[
            E(X) = \sum^\infty_{k=0} k q ^k p = \frac{q}{p}
        \]
    \end{example}
    \begin{block}{Proof}
        \[
            \begin{gathered}
                \sum^\infty_{k=0} q^k = \frac{1}{1-q}\\
                \sum^{\infty}_{k=0} kq^{k-1} =\frac{1}{(1-q)^2}\quad \text{(derivative both sides)}\\
                \therefore E(X) = \sum^{\infty}_{k=0} k q^k p = pq \sum^\infty_{k=0} k q^{k-1} = pq \frac{1}{(1-q)^2} = \frac{q}{p}
            \end{gathered}
        \]
    \end{block}
\end{frame}

\begin{frame}{Expectation}
    \begin{definition}[Negative Binomial distribution]
        In a sequence of independent Bernoulli trials with success probability $p$, if $X$ is the number of \ti{failures} before the $r$th success, then $X$ is said to have the \tb{Negative Binomial distribution} with parameters with parameters $r$ and $p$, denoted $X \sim \text{NBin}(r, p)$\newline

        if $X \sim NBin(r,p)$, then, the PMF of $X$ is
        \[
            P(X=n) = \binom{n+r-1}{r-1} p^r q^n
        \]
        for $n=0,1,2, \dots, $ where $q=1-p$
    \end{definition}
\end{frame}

\begin{frame}{Expectation}
    \begin{example}[Negative Binomial expectation]
        Let $X \sim \text{NBin}(r,p)$. Then $X = X_1 + \cdots + X_r$, where the $X_i$ are i.i.d. $Geom(p)$. By linearity, 
        \[
            E(X) = E(X_1) + \cdots + E(X_r) = r \cdot \frac{q}{p}
        \]
    \end{example}
\end{frame}

\begin{frame}{Expectation}
    Indicator r.v.  $I_A$ or ($I(A)$) for an event $A$ is defined to be $1$ if $A$ occurs and $0$ otherwise. So $I_A \sim \text{Bern}(p)$. 
    \begin{theorem}[Indicator r.v. properties]
        Let $A$ and $B$ be events. Then the following properties hold
        \begin{enumerate}
            \item $(I_A)^k = I_A$ for any positive integer $k$
            \item $I_{A^c} = 1 - I_A$
            \item $I_{A \cap B} = I_A I_B$
            \item $I_{A \cup B} = I_A + I_B - I_A I_B$
        \end{enumerate}
    \end{theorem}
\end{frame}

\begin{frame}{Expectation}
    \begin{theorem}[Fundamental bridge between probability and expectation]
        There is a \ti{one-to-one} correspondence between events and indicator r.v.s, and the probability of an event $A$ is the expected value of its indicator r.v. $I_A$: 
        \[
        P(A) = E(I_A)
        \]
    \end{theorem}

    Proof. For any event $A$, we have an indicator r.v. $I_A$. This is a one-to-one correspondence since $A$ uniquely determines $I_A$ and vice versa (to get from $I_A$ back to $A$, we can use the fact that $A = \{s \in S: I_A(s) = 1\}$). Since $I_A \sim \text{Bern}(p)$ with $p = P(A)$, we have $E(I_A) = P(A)$
\end{frame}

\begin{frame}{Expectation}
    \begin{theorem}[Expectation via survival function] \label{thm:expectation_via_survival_function}
        Let $X$ be a nonnegative integer-valued r.v. Let $F$ be the $CDF$ of $X$, and $G(x)=1-F(x)=P(X>x)$. The function $G$ is called the \tb{survival function} of $X$. Then
        \[
            E(X) = \sum^\infty_{k=0} G(n)
        \]
        That is, we can obtain the expectation of $X$ by summing up the survival function (or \ti{tail probability of the distribution})
    \end{theorem}
    Proof. We can represent $X$ as a sum of indicator r.v.s.: $X = I_1 + I_2 + \cdots + I_b$, where $I_n = I(X \geq n)$. (ex: if $X=5$, then $I_1, \cdots, I_5 = 1$ and $I_6, \cdots = 0$) Then
    \[
        E(X) = \sum^b_{k=1} E(I_k) = \sum^b_{k=1} P(X \geq k ) = \sum^{b-1}_{n=0} P(X > n) = \sum^\infty_{n=0} G(n)
    \]
\end{frame}

\begin{frame}{Expectation}
    \begin{example}[Rolling dice expectation via survival function]
        Suppose $X$ is the r.v. that represents the number of rolls of a fair die. Then $E(X) = \sum^\infty_{n=0}G(n) = \frac{(6+5+4+3+2+1+0)}{6}$
    \end{example}
    \begin{example}[Geometric expectation redux]
        Let $X \sim \text{Geom}(p)$, and $q = 1-p$. Using the Geometric story, $\{X > n\}$ is the event that the first $n+1$ trials are all failures. So by \cref{thm:expectation_via_survival_function},
        \[
            E(X) = \sum^\infty_{n=0} P(X >n) = \sum^\infty_{n=0} q^{n+1}= \frac{q}{1-q} = \frac{q}{p}
        \]
    \end{example}
    Compare procedures with \cref{ex:geometric_expectation}.
\end{frame}

\begin{frame}{Expectation}
    \begin{itemize}
        \item In general, $E(g(X)) \neq g(E(X))$ for arbitrary function $g$. 
        \item Therefore to find $E(g(X))$, first we have to know the distribution of $g(X)$ and apply the definition of expectation, $E(g(X))$.
        \item But what if $g(x)$ is difficult to determine? or hard to calculate? 
        \item Perhaps surprisingly, by $LOTUS$, without having to find the distrubition of $g(X)$, it is possible to find $E(g(X))$.
    \end{itemize}
    \begin{theorem}[\tb{Law of the unconscious statistician (LOTUS)}]
        If $X$ is a discrete r.v. and $g$ is a function from $\mbb{R}$ to $\mbb{R}$, then
        \[
            E[g(X)] = \sum_x g(x)P(X = x)
        \]
    \end{theorem}
\end{frame}

\begin{frame}
    (LOTUS) Let $X$ have support $0, 1, 2, \cdots$ with probabilities $p_0, p_1, p_2, \dots$ so the PMF is $P(X=n)=p_n$. Then $X^3$ has support $0^3, 1^3, 2^3, \dots$ with probabilities $p_0, p_1, P-2, \dots$, So
    \[
    \begin{gathered}
        E(X) = \sum^\infty_{n=0} np_n,\\
        E(X^3) = \sum^\infty_{n=0} n^3 p_n,
    \end{gathered}
    \]

\end{frame}

\begin{frame}{Expectation}
    LOTUS Proof.(skip)
    \[
    \begin{aligned}
        E(g(X)) &= \sum_s g(X(s))P(\{s\}) \\
        &= \sum_x \sum_{s:X(s)=x} g(X(s)) P(\{s\})\\
        &= \sum_x g(x) \sum_{s:X(s)=x} P(\{s\}) \\
        &= \sum_x g(x) P(X=x)
    \end{aligned}
    \]
\end{frame}

\begin{frame}{Expectation}
    One important application of LOTUS is for finding the \tb{variance} of a random variable.
    \begin{definition}[Variance and standard deviation]
        The \tb{variance} of an r.v. $X$ is
        \[ Var(X) = E(X-EX)^2\]
        THe square root of the variance is called the \tb{standard deviation} (SD)
        \[SD(X) = \sqrt{Var(X)}\]
    \end{definition}
\end{frame}

\begin{frame}{Expectation}
    \begin{theorem}
        For any r.v. $X$,
        \[
            Var(X) = E(X^2) - (EX)^2
        \]
    \end{theorem}
    Proof. Let $\mu = EX$. Then
    \[
        E(X-\mu)^2 = E(X^2 - 2\mu X + \mu^2) = E(X^2) - 2\mu EX +\mu^2 = E(X^2) - \mu^2. \qed
    \]
\end{frame}

\begin{frame}{Expectation}
    Variance has the following properties.
    \begin{itemize}
        \item $Var(X+c) = Var(X)$ for any constant $c$ (By def. of variance).
        \item $Var(cX) = c^2 Var(X)$ for any constnat $c$ (By def. of variance).
        \item If $X$ and $Y$ are independent, then $Var(X+Y) = Var(X) + Var(Y)$ (Proof in chapter 7).
        \begin{itemize}
            \item If $X$ and $Y$ are dependent, then equation is not hold.
            \item For example, if $X$ is always equals $Y$, we have \[
                Var(X+Y) = Var(2X) = 4Var(X) \neq 2Var(X) = Var(X) + Var(Y)
            \]
        \end{itemize}
        \item $Var(X) \geq 0$, with equality if and only if $P(X=a)=1$ for some constant $a$ (By def. of variance).
    \end{itemize}
\end{frame}


\begin{frame}{Expectation}
    \begin{example}[Geometric variance]
        Let $X \sim \text{Geom}(p)$. We already know $E(X)=\frac{q}{p}$. By LOTUS,
        \[
        \begin{gathered}
            E(X^2) = \sum^\infty_{k=0} k^2 P(X=k)=\sum^\infty_{k=0} k^2 p q^k = \sum^\infty_{k=0} k^2 p q^k = \sum^{\infty}_{k=1} k^2 p q^k \\
            = pq \frac{1+q}{(1-q)^3} = \frac{q(1+q)}{p^2}
        \end{gathered}
        \].
    \end{example}
    Use fact that 
    \[
        \begin{gathered}
            \sum^\infty_{k=0} q^k = \frac{1}{1-q}\quad
            \sum^\infty_{k=1}kq^{k-1} = \frac{1}{(1-q)^2}\\
            \sum^\infty_{k=1}kq^k = \frac{q}{(1-q)^2}\quad
            \sum^\infty_{k=1}k^2 q^{k-1} = \frac{1+q}{(1-q)^3}
        \end{gathered}
    \]
\end{frame}

\begin{frame}{Expectation}
    \begin{example}[Geometric Variance (continuous)]
        \[
        Var(X) = E(X^2) - (EX)^2 = \frac{q(1+q)}{p^2} - \frac{q}{p}^2 = \frac{q}{p^2}
        \].
    \end{example}
\end{frame}



\begin{frame}{Expectation}
    \begin{example}[Binomial variance]
        Let $X \sim \text{Bin}(n,p)$ using indicator r.v.s to avoid tedious sums, $X=I_1+I_2+\cdots+I_n$, where $I_j$ is the indicator of the $j$th trial being a success. Each $I_j$ has variance
        \[
        Var(I_j) = E(I_j^2) - (E(I_j))^2 = p - p^2 = p(1-p)
        \]
        Since the $I_j$ are independent, we can add their variances to get the variance of their sum:
        \[
        Var(X) = Var(I_1) + \cdots + Var(I_n) = np(1-p)
        \]
    \end{example}
\end{frame}

\begin{frame}{Expectation}
    \tb{Poission distribution} is an extremely popular \tb{discrete} distribution for modeling \ti{discrete data}.
    \begin{itemize}
        \item Note that $\sum^\infty_{k=0} \frac{\lambda^k}{k!}$ by taylor series.
    \end{itemize}
    \begin{definition}[Poission distribution]
        An r.v. $X$ has the \tb{Poission distribution} with parameter $\lambda$, where
        $\lambda > 0$, if the PMF of $X$ is
        \[P(X=k)=\frac{e^{-\lambda}\lambda^k}{k!},\quad k=0,1,2,\dots\]
        It is denoted by $X \sim \text{Pois}(\lambda)$.
    \end{definition}
    Poission distribution is a valid PMF because $\sum^\infty_{k=0} \frac{e^{-\lambda}\lambda^k}{k!} = e^{-\lambda} \sum^\infty_{k=0} \frac{\lambda^k}{k!} = e^{-\lambda}e^{\lambda} = 1$
\end{frame}

\begin{frame}{Expectation}
    \begin{example}[Poisson expectation]
        Let $X \sim \text{Pois}(\lambda)$. We will show that the mean and variance are both equal to $\lambda$. For the mean, we have 
        \[
        \begin{aligned}
            E(X) &= e^{-\lambda} \sum^\infty_{k=0} k \frac{\lambda^k}{k!} \\
            &= e^{-\lambda} \sum^\infty_{k=1} k\frac{\lambda^k}{k!} \\
            &= \lambda e^{-\lambda} \sum^\infty_{k=1} \frac{\lambda^{k-1}}{(k-1)!} \\
            &= \lambda e^{-\lambda} e^{\lambda} = \lambda
        \end{aligned}
        \]
    \end{example}
\end{frame}

% \begin{frame}
%     \begin{example}[Poisson variance]
%         Let $X \sim \text{Pois}(\lambda)$. We will show that the mean and variance are both equal to $\lambda$. For the variance, we have 
%         \[
%         \begin{aligned}
%             E(X^2) &= e^{-\lambda} \sum^\infty_{k=0} k^2 \frac{\lambda^k}{k!} \\
%             &= e^{-\lambda} \sum^\infty_{k=1} k^2\frac{\lambda^k}{k!} \\
%             &= \lambda e^{-\lambda} \sum^\infty_{k=1} k\frac{\lambda^{k-1}}{(k-1)!} \\
%             &= \lambda e^{-\lambda} e^{\lambda} = \lambda
%         \end{aligned}
%         \]
%         Therefore, $Var(X) = E(X^2) - (EX)^2 = \lambda$
%     \end{example}

% \end{frame}

\begin{frame}
    \begin{example}[Exercise 4.43]
        A building has $n$ floors, labeled $1, 2, \dots, n$. At the first floor, $k$ people enter the elevator, which is going up and is empty before they enter. Independently, each decides which of floors $2, 3, \dots, n$ to go to and presses that button (unless someone has already pressed it).
        \begin{enumerate}
            \item[a] Assume for this part only that the probabilities for floors $2, 3, \dots, n$ are equal. Find the expected number of stops the elevator makes on floors $2, 3, \dots , n$.
        \end{enumerate}    
    \end{example}
    \ti{Solution(a):} Let $X$ be the r.v. that represents number of floors on which elevator stops, and $I_i$ be the indicator r.v. that specific floor number of stop. Then $X = I_1+I_2+\dots +I_n$. $E[X] = E[I_2] +\dots +E[I_n] =P(I_2=1) +\dots + P(I_n=1)$. $P(I_i=1) = 1 - P(\text{no one among the }k\text{ press }i) = 1- \left(\frac{n-2}{n-1} \right)^k$. $\therefore E[X] = (n-1) \left[1-\left(\frac{n-2}{n-1}\right)^k\right]$.
\end{frame}

\end{document}