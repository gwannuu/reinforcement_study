\documentclass[8pt]{beamer}
\usefonttheme[onlymath]{serif}


\setbeamertemplate{frametitle}{%
  \vskip1ex
  \usebeamerfont{frametitle}%
  \insertsubsectionhead\par        %  ← 원하는 대로 변경 가능
  \vskip1ex
  \hrule                             % 밑줄(선택)
}

% 테마 선택 (선택 사항)
% \usetheme{Madrid} % 기본 테마, 다른 테마 사용 가능
% \font{serif}
\usepackage{amsfonts}
\usepackage{amssymb}
\usepackage[T1]{fontenc} % To use combination of textbf, textit
\usepackage[dvipsnames]{xcolor}   % can use more variant colors

% \setcounter{MaxMatrixCols}{20}

% (필요한 패키지들)
% \usepackage{amsthm}
\setbeamertemplate{theorems}[numbered]  % 정리, 정의 등에 번호를 달아줌

% \theoremstyle{plain} % insert bellow all blocks you want in italic
% \newtheorem{theorem}{Theorem}[section] % to number according to section
% 
% \theoremstyle{definition} % insert bellow all blocks you want in normal text
% \newtheorem{definition}{Definition}[section] % to number according to section
% \newtheorem*{idea}{Proof idea} % no numbered block

\newtheorem{proposition}[theorem]{Proposition}

\usepackage{tcolorbox}

% 필요할 경우 패키지 추가
\usepackage{graphicx} % 이미지 삽입을 위한 패키지
\usepackage{amsmath}   % 수식 사용
\usepackage{hyperref}  % 하이퍼링크 추가
\usepackage{cleveref}
\usepackage{multicol}  % 여러 열 나누기
\usepackage{ulem} % 취소선 및줄 나누기



\newcommand{\mrm}[1]{\mathrm{#1}}
\newcommand{\mbb}[1]{\mathbb{#1}}
\newcommand{\mb}[1]{\mathbf{#1}}
\newcommand{\mc}[1]{\mathcal{#1}}
\newcommand{\tb}[1]{\textbf{#1}}
\newcommand{\ti}[1]{\textit{#1}}
\newcommand{\mypois}[1]{\operatorname{Pois}(#1)}

\newcommand{\myber}[1]{\operatorname{Bern}\!\left(#1\right)}
\newcommand{\mybin}[2]{\operatorname{Bin}\!\left(#1,#2\right)}
\newcommand{\mytoinf}[1]{#1 \rightarrow \infty}
\newcommand{\myexp}[1]{\exp{\left(#1\right)}}
\newcommand{\myunif}[2]{\operatorname{Unif}\!\left(#1, #2\right)}
\newcommand{\mygeom}[1]{\operatorname{Geom}\!\left(#1\right)}
\newcommand{\Expo}[1]{\operatorname{Expo}\!\left(#1\right)}
\newcommand{\abs}[1]{\left\lvert #1 \right\rvert}
\newcommand{\expec}[1]{\operatorname{E}\left[ #1 \right]}
\newcommand{\Var}[1]{\operatorname{Var}\left[#1\right]}
\newcommand{\myskew}[1]{\operatorname{Skew}\!\left[#1\right]}
\newcommand{\mykurt}[1]{\operatorname{Kurt}\!\left[#1\right]}
\newcommand{\mywei}[2]{\operatorname{Wei}\!\left(#1, #2\right)}
\newcommand{\Span}[1]{\operatorname{Span}\!\left(#1\right)}
\newcommand{\Cov}[2]{\operatorname{Cov}\!\left(#1, #2\right)}
\newcommand{\intinfty}{\int_{-\infty}^\infty}
\newcommand{\Corr}[2]{\operatorname{Corr}\!\left(#1, #2\right)}

% 발표 제목, 저자, 날짜 설정
\title{Probability: Joint Distributions}
\author{Gwanwoo Choi}
% \date{}

\begin{document}
% 표지 슬라이드
\begin{frame}
    \titlepage
\end{frame}

% % 목차 슬라이드
% \begin{frame}
%     \frametitle{Table of Contents}
%     \tableofcontents
% \end{frame}

\subsection{Joint, marginal and conditional}

\begin{frame}{.}
    \begin{definition}[Joint PDF]
        The \ti{joint} CDF of r.v.s $X$ and $Y$ is the function $F_{X,Y}$ given by
        \[
            F_{X,Y} (x,y) = P(X\leq x, Y \leq y)
        \]
        The joint CDF of $n$ r.v.s is defined analogously
    \end{definition}

    \begin{definition}[Joint PMF]
        The \ti{joint} PMF of discrete r.v.s $X$ and $Y$ is the function 
        \[p_{X,Y}(x,y) = P(X=x,Y=y)\]
        The joint PMF of $n$ r.v.s is defined analogously
    \end{definition}

    \begin{itemize}
        \item for all $x$ and $y$,  $0 \leq P(X=x,Y=y) \leq 1$
        \item $\sum_x \sum_y P(X=x,Y=y) = 1$
    \end{itemize}
\end{frame}

\begin{frame}{.}
    \begin{definition}[Marginal PMF]
        For discrete r.v.s $X$ and $Y$, the \ti{marginal} PMF of $X$ is 
        \[
            P(X=x) = \sum_y P(X=x,Y=y)
        \]
    \end{definition}
    Another way to obtain marginal PMF is using the joint CDF
    \[
        F_X(x) = P(X \leq x) = \lim_{y \rightarrow \infty} P(X\leq x, Y\leq y) = \lim_{y \rightarrow \infty} F_{X,Y} (x,y)
    \]

    \begin{definition}[Conditional PMF]
        For discrete r.v.s $X$ and $Y$, the \ti{conditional} PMF of $Y$ given $X = x$ is 
        \[
        P(Y=y |X=x) = \frac{P(X=x,Y=y)}{P(X=x)}
        \]
    \end{definition}

    Note that by Bayes' rule:
    \[
    P(Y=y|X=x) = \frac{P(X=x|Y=y) P(Y=y)}{P(X=x)}
    \]
\end{frame}

\begin{frame}{.}
    \begin{definition}[Independence of r.v.s]
        Random variables $X$ and $Y$ are \ti{independent} if for all $x$ and $y$,
        \[
            F_{X,Y} (x,y) = F_X(x) F_Y(y)
        \]
        If $X$ and $Y$ are discrete, this is equivalent to the conditions
        \[
        \begin{gathered}
            P(X=y,Y=y) = P(X=x)P(Y=y)\\
            P(X=x|Y=y) = P(X=x) \quad (P(Y=y) > 0)
        \end{gathered}
        \]
    \end{definition}
    \begin{example}[Chicken-egg]
        Suppose a chicken lays a random number of eggs, $N$, where $N\sim \mypois{\lambda}$. Each egg independently hatches with probability $p$ and fails to hatch with probability $q= 1 - p$. Let $X$ be the number of eggs that hatch and $Y$ the number that do not hatch, so $X+Y = N$. What is the joint PMF of $X$ and $Y$?
    \end{example}

    $X|N \sim \mybin{n}{p}$ and $Y|N \sim \mybin{n}{q}$. $P(X=i,Y=j) = \sum_{n=0}^\infty P(X=i,Y=j|N=n)P(N=n) = P(X=i,Y=j) = P(X=i,Y=j|N=i+j)P(N=i+j) = P(X=i|N=i+j)P(N=i+j)$.

    $P(X=i,Y=j) = P(X=i|N=i+j)P(N=i+j) = \binom{i+j}{i} p^i q^j e^{-\lambda} \lambda^{i+j} / (i+j)! = (e^{-\lambda p} (\lambda p)^i / i!)\cdot (e^{-\lambda q} (\lambda q)^{j} / j!) \implies X \sim \mypois{\lambda p}, Y \sim \mypois{\lambda q}$
\end{frame}

\begin{frame}{.}
    \begin{theorem}
        If $X\sim \mypois{\lambda p}, Y\sim \mypois{\lambda q}$, then $N=X+Y \sim \mypois{\lambda}$ and $X|N \sim \mybin{n}{p}$
    \end{theorem}
    \begin{theorem}
        If $N \sim \mypois{\lambda}$ and $X|N = n \sim \mybin{n}{p}$, then $X\sim \mypois{\lambda p }, Y = N-X \sim \mypois{\lambda q}$, and $X$ and $Y$ are independent
    \end{theorem}
    \begin{definition}[Joint PDF]
        If $X$ and $Y$ are continuous with joint CDF $F_{X,Y}$, their \ti{joint} PDF is the derivative of the joint CDF with respect to $x$ and $y$
        \[
            f_{X,Y}(x,y) = \frac{\partial^2}{\partial x \partial y} F_{X,Y}(x,y)
        \]
        $f_{X,Y} (x,y)$ should satisfy
        \begin{itemize}
            \item $f_{X,Y} (x,y) \geq 0$
            \item $\int_{-\infty}^\infty \int_{-\infty}^\infty f_{X,Y}(x,y) dx dy =1$
        \end{itemize}
    \end{definition}
    For a general region $A \subseteq \mbb{R}^2$, $P((X,Y)\in A) = \iint_{A} f_{X,Y} (x,y) dx dy$
\end{frame}

\begin{frame}{.}
    \begin{definition}[Marginal PDF]
        For continuous r.v.s $X$ and $Y$ with joint PDF $f_{X,Y}$, the \ti{marginal} PDF of $X$ is 
        \[
            f_X(x) = \int_{-\infty}^\infty f_{X,Y} (x,y) dx dy
        \]
    \end{definition}

    \begin{definition}[Conditional PDF]
        For continuous r.v.s $X$ and $Y$ with joint PDF $f_{X,Y}$, the \ti{conditional} PDF of $Y$ given $X=x$ is 
        \[
        \begin{cases}
            f_{Y|X}(y|x) = \frac{f_{X,Y}(x,y)}{f_X(x)}  = \frac{f_{X,Y}(x,y)}{\int_{-\infty}^\infty f_{X,Y}(x,y) dy} &\quad (f_X(x) > 0) \\ 
            f_{Y|X}(y|x) = 0 &\quad (f_X(x) = 0)
        \end{cases}
        \]
    \end{definition}
\end{frame}

\begin{frame}{.}
    \begin{theorem}[Continuous form of Bayes' rule and LOTP]
        For continuous r.v.s $X$ and $Y$, we have the following continuous form of Bayes' rule:
        \[
            f_{Y|X}(y|x) = \frac{f_{X|Y}(x|y)f_Y(y)}{f_X(x)} \quad (f_X(x)>0)
        \]
        And we have the following continuous form of the law of total probability
        \[
        f_X(x) = \int_{-\infty}^\infty f_{X|Y}(x|y) f_Y(y)dy
        \]
    \end{theorem}

    \begin{definition}[Independence of continuous r.v.s]
        Random variables $X$ and $Y$ are \ti{independent} if for all $x$ and $y$,
        \[
        F_{X,Y} (x,y) = F_X(x)F_Y(y)
        \]
        If $X$ and $Y$ are continuous with joint PDF $f_{X,Y}$, this is equivalent to
        \[
        \begin{gathered}
            f_{X,Y} = f_X(x)f_Y(y) \\
            f_{Y|X} = f_Y(y) \quad (f_X(x) > 0)
        \end{gathered}
        \]
    \end{definition}

    
\end{frame}

\begin{frame}{.}
    \begin{proposition}
        Suppose the joint PDF $f_{X,Y}$ of $X$ and $Y$ factors as 
        \[
        f_{X,Y}(x,y) = g(x) h(y)
        \]
        for all $x$ and $y$, where $g$ and $h$ are nonnegative functions.
        Then $X$ and $Y$ are independent.
    \end{proposition}
    \begin{proof}
        \[
    \begin{gathered}
        f_X(x) = g(x) \int_{-\infty}^\infty h(y) dy \\
        f_Y(y) = h(y) \int_{-\infty}^\infty g(x) dx \\
        f_X(x) f_Y(y) = g(x) h(y) \iint_{-\infty}^\infty g(x) h(y) dx dy = g(x) h(y) = f_{X,Y} (x,y)
    \end{gathered}
    \]
    \end{proof} 

    Note that $f_X(x) = \frac{g(x)}{\int_{-\infty}^\infty g(x)dx}$ and $f_Y(y) = \frac{h(y)}{\int_{-\infty}^\infty h(y) dy}$

\end{frame}

\begin{frame}{.}
    \begin{example}
    Let $(X,Y)$ be a completely random point in the unit disk $\{(x,y): x^2 + y^2 \leq 1\}$ with PDF 
    \[
    f_{X,Y}(x,y) =
    \begin{cases}
        \frac{1}{\pi}  & \text{if } x^2 + y^2 \leq 1\\
        0 & \text{otherwise}
    \end{cases}
    \]
    \begin{itemize}
        \item $X$ and $Y$ are not independent since $f_{X,Y}(\frac{\sqrt{3}}{2}, \frac{\sqrt{3}}{2}) = 0$ and $f_Y(\frac{\sqrt{3}}{2})f_Y(\frac{\sqrt{3}}{2})\neq 0$.
        \item $X$ and $Y$ are not independent, since larger value $\abs{X}$ restricts $Y$ to be a smaller range
        \item $f_X(x) = \int_{-\sqrt{1 - x^2}}^{\sqrt{1 - x^2}}\frac{1}{\pi} dy= \frac{2}{\pi} \sqrt{1 - x^2}, -1 \leq x \leq 1$. $f_{Y|X}(y|x) = \frac{f_{X,Y}(x,y)}{f_X(x)} = \frac{1/\pi}{2\sqrt{1 - x^2} /\pi} = \frac{1}{2\sqrt{1 - x^2}}, -\sqrt{1 - x^2} \leq y \leq \sqrt{1 - x^2}$ $\implies$ the fact that conditional PDF $f_{Y|X}(y|x)$ is not free of $x$ means $Y$ is not independent with $X$.
    \end{itemize}
    \end{example}
\end{frame}

\begin{frame}{.}
    \begin{example}[Comparing Exponentials of different rates]
        Let $T_1 \sim \Expo{\lambda_1}$ and $T_2 \sim \Expo{\lambda_2}$ be independent. 
        Find $P(T_1 < T_2)$. For example, $T_1$ could be the lifetime of a refrigerator and $T2$ could be the lifetime of a stove (if we are willing to assume Exponential distributions for these), and then $P(T_1<T_2)$ is the probability that the refrigerator fails before the stove. 
        We know that $\min{T_1, T_2} \sim \Expo{\lambda_1 + \lambda_2}$, which tells us about \ti{when} the first appliance failure will occur, but we may also want to know about \ti{which} appliance will fail first.
    \end{example}
    \[
        \begin{aligned}
            P(T_1< T_2) &= \int_{0}^\infty \int_0^{t_2} \lambda_1e^{-\lambda_1 t_1} \lambda_2 e^{-\lambda_2 t_2} dt_1 dt_2 
            = \int_0^\infty (1 - e^{-\lambda_1 t_2}) \lambda_2 e^{-\lambda_2 t_2} dt_2 \\
            &= 1 - \lambda_2\int_0^\infty e^{-(\lambda_1 + \lambda_2)t_2} dt_2 = 1- \frac{\lambda_2}{\lambda_1 + \lambda_2} = \frac{\lambda_1}{\lambda_1 + \lambda_2}
        \end{aligned}
    \]
\end{frame}

\begin{frame}{.}
    \begin{example}[Cauchy PDF]
        Let $X$ and $Y$ be i.i.d. $\mathcal{N}(0,1)$, and let $T = X/Y$. (We can define $T$ arbitrary in the case $Y=0$; the choice of how to define $T$ in that case has no effect on the distribution of $T$, since $P(Y=0)=0$.) The distribution of $T$ is a famous named distribution called the \tb{Cauchy} distribution. Find the PDF of $T$.
    \end{example}
    Note that $P\left(\frac{X}{Y} \leq t\right) = P(X \leq tY, Y>0) + P(X \geq tY, Y<0) = P(X\leq tY, Y>0) + P(X \leq -tY, Y<0) = P(X\leq t\abs{Y})$.
    \[
    \begin{aligned}
        F_T(t) &= P(T\leq t) = P\left(\frac{X}{Y} \leq t\right) = P(X \leq tY, Y>0) + P(X \geq tY, Y<0) \\
        &= \int_0^\infty \int_{-\infty}^{ty} \frac{1}{\sqrt{2\pi}} e^{-x^2/2}  dx \frac{1}{\sqrt{2\pi}}e^{-y^2/2}dy \\
        &+ \int_{-\infty}^0 \int_{ty}^{\infty} \frac{1}{\sqrt{2\pi}}e^{-x^2/2} dx \frac{1}{\sqrt{2\pi}} e^{-y^2/2} dy \\ 
        &= \int_{-\infty}^\infty \int_{-\infty}^{t\abs{y}} \frac{1}{\sqrt{2\pi}}e^{-x^2/2} dx \frac{1}{\sqrt{2\pi}} e^{-y^2/2} dy = \frac{2}{\sqrt{2\pi}}\int_0^\infty \Phi(ty)  e^{-y^2/2} dy
    \end{aligned}
    \]
\end{frame}

\begin{frame}{.}
    \[
    \begin{aligned}
        f_T(t) &= F_T^\prime(t) = \frac{2}{\sqrt{2\pi}} \int_0^\infty \frac{\partial}{\partial t} \Phi(ty) e^{-y^2/2} dy \\
        &=  \frac{2}{\sqrt{2\pi}} \int_0^\infty y \frac{1}{\sqrt{2\pi}}e^{-(ty)^2/2} e^{-y^2/2} dy = \frac{1}{\pi} \int_0^\infty y e^{-(t^2 +1)y^2/2} dy \\
        &= \frac{1}{\pi (1+t^2)} \left[ e^{-(t^2 +1)y^2/2} \right]^0_\infty = \frac{1}{\pi (1+t^2)}
    \end{aligned}
    \]
    And $F_T(t) = \int_{-\infty}^t \frac{1}{\pi(1+t^2)} = \frac{1}{\pi} \operatorname{arctan}(t) + \frac{1}{2}$.

    An interesting fact about cauchy distribution is that although the PDF is symmetric about $0$, its expected value does not exist, since the integral $\int_{-\infty}^\infty \frac{t}{\pi(1+t^2)}dt$ diverges. For large $t$, $\frac{t}{1+t^2} \approx \frac{1}{t}$, and $\int^\infty_1 \frac{1}{t} dt = \infty$

\end{frame}

\begin{frame}{.}

    Suppose $X$ is \ti{discrete} r.v and $Y$ is \ti{continuous} r.v. Then \ti{marginal} distribution is defined by
    \[
        \begin{gathered}
            P(X=x) = \int_{-\infty}^\infty f_{X,Y}(X=x,y) dy\\
            f_Y(y) = \sum_x f_{X,Y}(X=x,y)
        \end{gathered}
    \]
    Conditional r.v. $X|Y$ and $Y|X$ are defined by
    \[
        \begin{gathered}
            P(X=x|y) = \frac{f_{X,Y}(X=x,y)}{\sum_x f_{X,Y} (X=x, y)} = \frac{f_{X,Y} (X=x,y)}{f_Y(y)} \\
            f_{Y|X} (y|X=x) = \frac{f_{X,Y}(X=x,y)}{\int_{-\infty}^\infty f_{X,Y}(X=x,y) dy} = \frac{f_{X,Y}(X=x,y)}{P(X=x)}
        \end{gathered}
    \]
    Bayes' rules are defined by 
    \[
        \begin{gathered}
            P(X=x|y) = \frac{f_{Y|X}(X=x,y) P(X=x)}{f_Y(y)}\\
            f_{Y|X}(y|X=x) = \frac{P(X=x|y)f_Y(y)}{P(X=x)}
        \end{gathered}
    \]
\end{frame}

\begin{frame}{.}
    \begin{example}
      A lightbulb was manufactured by one of two companies. 
      Bulbs that are made by Company 0 last an $\Expo{\lambda_0}$ amount of time, and bulbs made by Company 1 last an $\Expo{\lambda_1}$ amount of time, with $\lambda_0 < \lambda_1$. 
      The bulb of interest here was made by Company 0 with probability $p_0$ and by Company 1 with probability $p_1 = 1 - p_0$, but from inspecting the bulb we don't know which company made it.

      Let $T$ be how long the bulb lasts, and $I$ be the indicator of it having been made by Company 1.
      \begin{enumerate}
        \item Find the CDF and PDF of $T$.
        \item Does $T$ have the memoryless property?
        \item Find the conditional distribution of $I$ given $T=t$. What happens to this as $t \rightarrow \infty$?
      \end{enumerate}
    \end{example}
    1. $F_{T}(t) = P(T\leq t) = P(T\leq t|I=0)P(I=0) + P(T\leq t|I=1)P(I=1) = p_0 (1 - e^{-\lambda_0 t} )   + p_1 ( 1- e^{\lambda_1 t} ) = 1 - p_0 e^{-\lambda_0 t} - p_1 e^{-\lambda_1 t}, \forall t>0$. 
    $f_T(t) = \lambda_0 p_0 e^{-\lambda_0 t} + \lambda_1 p_1 e^{-\lambda_1 t}, \forall t>0$.

    2. No. because the distribution of $T$ is not Exponential.

    3.$P(I=1|t) = \frac{f_{T|I}(t|I=1) P(I=1)}{f_T(t)} = \frac{\lambda_1 p_1 e^{-\lambda_1 t}}{\lambda_0 p_0 e^{-\lambda_0 t} + \lambda_1 p_1 e^{-\lambda_1 t}}$. $t\rightarrow \infty$, then $P(I=1|t) \rightarrow 0$, which implies that the longer the bulb lasts, the more confident we will be that it was made by company 0
\end{frame}

\subsection{2D Lotus}

\begin{frame}{.}
    \begin{theorem}[2D LOTUS]
        Let $g$ be a function that $g: \mbb{R}^2 \rightarrow \mbb{R}$. If $X$ and $Y$ are discrete, then 
        \[
            \expec{g(X,Y)} = \sum_x \sum_y g(x,y) P(X=x,Y=y)
        \]
        If $X$ and $Y$ are continuous with joint PDF $f_{X,Y}$, then
        \[
            \expec{g(X,Y)} = \int_{-\infty}^\infty \int_{-\infty}^\infty g(x,y) f_{X,Y} (x,y) dx dy
        \]
    \end{theorem}

    \begin{example}
        Let $X$ and $Y$ be i.i.d. $\myunif{0}{1}$ r.v.s. Find $\expec{\abs{X-Y}}$
    \end{example}
    \[
    \begin{aligned}
        \expec{\abs{X-Y}} &= \int_0^1 \int_0^1 \abs{x-y} dx dy = \int_0^1 \int_y^1 (x-y)dx dy + \int_0^1 \int_0^y (y-x) dx dy \\
        &= 2 \int_0^1 \int_y^1 (x-y) dx dy = \frac{1}{3}
    \end{aligned}
    \]

\end{frame}

\begin{frame}{.}
    \begin{example}
        For $X, Y \overset{\text{i.i.d}}{\sim} \mc{N}(0,1)$. Find $\expec{\abs{X-Y}}$.
    \end{example}
    \[
    \expec{\abs{X-Y}} = \int_{-\infty}^\infty \int_{-\infty}^\infty \abs{x-y} \frac{1}{2\pi} e^{-(x^2 +y^2)/2}dxdy
    \]
    Calculate this. or another simple solution is using $X-Y = \sqrt{2}Z \sim \mc{N}(0, 2) (\because X+Y \sim \mc{N}(\mu_1 + \mu_2, \sigma_1^2 + \sigma_2^2)) $.

    \[
        \expec{\abs{X-Y}} = \sqrt{2}\expec{\abs{Z}} = \frac{2}{\pi}
    \]

    \begin{proposition}[Linearity via 2D LOTUS]
        Let $X$ and $Y$ be continuous r.v.s (the analogous method also works in the discrete case). By 2D LOTUS,
        \[
        \begin{aligned}
            \expec{(X+Y)} &= \int_{-\infty}^\infty (x+y)f_{X,Y}(x,y)dx dy \\
            &=\int_{-\infty}^\infty \int_{-\infty}^\infty x f_{X,Y}(x,y) dxdy + \int_{-\infty}^\infty \int_{-\infty}^\infty y f_{X,Y} dxdy \\ 
            &= \int^{\infty}_{-\infty} x \int_{-\infty}^\infty f_{X,Y} (x,y) dy dx + \int_{-\infty}^\infty y \int^{\infty}_{-\infty} f_{X,Y} (x,y) dx dy \\
            &= \expec{X} + \expec{Y}
        \end{aligned}
        \]
    \end{proposition}
\end{frame}
\subsection{Covariance and correlation}

\begin{frame}{.}
    \begin{definition}[Covariance]
        The \tb{covariance} between r.v.s $X$ and $Y$ is 
        \[
            \Cov{X}{Y} = \expec{(X - \expec{X})(Y - \expec{Y})} = \expec{XY} - \expec{X}\expec{Y}
        \]
        \begin{itemize}
            \item If $X$ and $Y$ are independent, then their covariance is zero. 
            \item We say that r.v.s with zero covariance are \tb{uncorrelated} $(\expec{XY} = \expec{X}\expec{Y})$.
        \end{itemize}
    \end{definition}

    \begin{theorem}
        If $X$ and  $Y$ are independent, then they are uncorrelated. (Converse is false)
    \end{theorem}

    \begin{proof}
        \[
        \begin{aligned}
            \expec{XY} &= \int_{-\infty}^\infty \int_{-\infty}^\infty xy f_{X,Y}(x,y) dx dy = \int_{-\infty}^\infty xy f_X(x)f_Y(y) dx dy \\
            &= \intinfty y f_Y(y) \left(\intinfty x f_X(x) dx \right) dy = \intinfty xf_X(x) dx \intinfty y f_Y(y) dy\\
            &=\expec{X}\expec{Y}
        \end{aligned}
        \]

        Converse is false. Let $Y = X^2, X \sim \mc{N}(0,1)$. Then $\expec{XY} = \expec{X^3} = 0$ and $\expec{x}\expec{y}= 0$. But $X$ and $Y$ are dependent.
    \end{proof}
\end{frame}

\begin{frame}{.}
    \begin{columns}
        \begin{column}{0.4\textwidth}
            Covariance is a measure of \tb{linear association}, so r.v.s can be dependent in nonlinear ways and still have zero covariance like in example $Y = X^2, X \sim \mc{N}(0,1)$
            \begin{itemize}
                \item $X$ and $Y$ is dependent in nonlinear ways and still have zero covariance.
                \item Bottom right plot of figure shows draws from the joint distribution of $X$ and $Y$.
            \end{itemize}
        \end{column}
        \begin{column}{0.6\textwidth}
            \begin{figure}
                \includegraphics[width=1.0\textwidth]{CorrelateFigures.png}
            \end{figure}
        \end{column}
    \end{columns}
\end{frame}

\begin{frame}{.}
    Covariance has the following key properties

    \begin{enumerate}
        \item $\Cov{X}{X} = \Var{X}$
        \item $\Cov{X}{Y} = \Cov{Y}{X}$
        \item $\Cov{X}{c} = 0, \forall c \in \mbb{R}$
        \item $\Cov{aX}{Y} = a\Cov{X}{Y}, \forall a \in \mbb{R}$
        \item $\Cov{X+Y}{Z}=\Cov{X}{Z}+\Cov{Y}{Z}$
        \item $\Cov{X+Y}{Z+W} = \Cov{X}{Z}+\Cov{Y}{Z}+\Cov{X}{W}+\Cov{Y}{W}$
        \item $\Var{X+Y} = \Var{X}+\Var{Y}+2\Cov{X}{Y}$
        $= \Cov{X+Y}{X+Y} = \Cov{X}{X}+ \Cov{Y}{Y} + 2\Cov{X}{Y}$
    \end{enumerate}

    \begin{proposition}
        For independent r.v.s, the variance of the sum is the sum of the variances
        \[
            \Var{\sum_{j=1}^n X_j} = \sum_{j=1}^n \Var{X_j} 
        \]
    \end{proposition}
    \[
    \Cov{\sum_{j=1}^n X_j}{\sum_{i=1}^n X_j} = \sum_{j=1}^n \sum_{i=1}^n\Cov{X_j}{X_i} = \sum_{j=1}^n \Cov{X_j}{X_j}
    \]
\end{frame}

\begin{frame}{.}
    \begin{definition}[Correlation]
        The \tb{correlation} between r.v.s $X$ and $Y$ is
        \[
            \Corr{X}{Y} = \frac{\Cov{X}{Y}}{\sqrt{\Var{X} \Var{Y}}}
        \]
        This is only defined when $\Var{X} \neq 0$ and $\Var{Y} \neq 0$.
    \end{definition}
    For any constant $c$,
        \[
            \Corr{cX}{Y} = \frac{c\Cov{X}{Y}}{\sqrt{c^2 \Var{X}}{\Var{Y}}} = \Corr{X}{Y}
        \]
\end{frame}

\begin{frame}{.}
    \begin{theorem}[Correlation bounds]
        For any r.v.s $X$ and $Y$,
        \[
        -1 \leq \Corr{X}{Y} \leq 1
        \]
    \end{theorem}
    \begin{proof}
        Let $X$ and $Y$ be r.v.s s.t. $\Var{X} =1$ and $\Var{Y} =1$. Let $\rho = \Corr{X}{Y} = \Cov{X}{Y}$
        \[
        \begin{aligned}
            \Var{X+Y} \geq 0 \implies \Var{X} + \Var{Y} + 2\rho \geq 0 \implies -1\leq \rho\\
            \Var{X-Y} \geq 0 \implies \Var{X}+\Var{Y} - 2\rho \geq 0 \implies \rho \leq 1 \\
            \therefore -1 \leq \rho \leq 1
        \end{aligned}
        \]
    \end{proof}
\end{frame}

\begin{frame}{.}
    \begin{example}[Exponential max and min]
        Let $X$ and $Y$ be i.i.d. $\Expo{1}$ r.v.s. Find the correlation between $\max{(X,Y)}$ and $\min{(X,Y)}$.
    \end{example}

    Let $M = \max{(X,Y)}$ and $L = \min{(X,Y)}$. $L \sim \Expo{2}$ because $P(L \leq t) = P(X < t) + P(Y\leq t) = 1 - P(X>t, Y>t) = 1 - e^{-t} e^{-t} = 1- e^{-2t}, P(L = t) = 2e^{-2t}\implies L \sim \Expo{2}$.

    By memoryless property, $M-L \sim \Expo{1}$.

    \[
        \Cov{M}{L} = \Cov{M-L + L}{L} = \Cov{M-L}{L} + \Var{L} = \Var{L} =\frac{1}{4}
    \]
    \[
        \Var{M} = \Var{M-L+L} = \Var{M-L} + \Var{L} = 1+ \frac{1}{4} = \frac{5}{4}
    \]

    \[
        \Corr{M}{L} = \frac{\Var{L}}{\sqrt{\Var{M}}\Var{L}} = \frac{\sqrt{\Var{L}}}{\sqrt{\Var{M}}} = \frac{1}{\sqrt{5}}
    \]

\end{frame}

\end{document}